\section{Theory}\label{sec:theory}

In general, the single most important element of a research paper in our field is the physical insight which it conveys. In some cases, this is achieved by relating the results of our laboratory measurements to the predictions of an underlying theory. In other cases, our data provide information on one or more parameters which are elements of a physical theory. An example of the former is our study of the photoelectric effect, in which Einstein's very simple formula relating the work function, the frequency of light, and the maximum energy of the ejected electron is tested against measured data. Alternately, our experiment to measure the charge on the electron provides the value of e, which is not predicted by any theory.

The Theory section of your paper should provide a clear description of the physical phenomenon you are studying. It will typically include one or a few important mathematical results. For example, a paper on the photoelectric project will cite Einstein's result,\cite{Marg_88}
\begin{equation}\label{eq:photoelectric_energy}
    T_{\text{max}}=hf-\phi,
\end{equation}
where $T_{\text{max}}$ is the electron's maximum kinetic energy, $h$ is Planck's constant, $f$ is the frequency of the light, and $\phi$ is the work function. Notice first that each formula is separated from the text and sequentially numbered, and that each symbol appearing in the equation is defined in the text. Equation 1 could be used as the starting point for your description of the physical mechanism of the photoelectric effect.

Another example is provided by the oil drop experiment. In that case, equations are used to relate the measured rise and fall times of the drops to their charge. You should not derive these results, but instead they should simply be stated, with a citation, in the Theory section of your paper.

Overall, the content of this section will form the basis for your subsequent interpretation of your measured data.