\section*{Appendix}

An appendix to a conventional journal article will typically contain either data in the form of one or more tables or figures, or additional explanatory text which may also include mathematical derivations. In our papers, only data tables and figures may be included, not text or mathematics.

When would you choose to put data in the Appendix, rather than in the Results section? Typically, data tables that appear as part of the Results are of limited length and complexity. The data we show in Table 1 are "not too long," and "not too complex." But if our data instead consisted of the mean terminal velocity of each of 10 oil drops in each of three electric field settings, then those results should appear in a table in the Appendix. Tables which appear in the Appendix should contain only a limited set of "distilled" data.

In some cases it may be useful to show one or more plots of the data, beyond those which appear in the Results. We may, for example, elect to include a figure showing "raw" data in the Appendix.

Each data table and figure in the Appendix must have a caption.