\section{Introduction} \label{sec:introduction}

The introductory section of your paper is intended to give the reader a clear picture of the scope and significance of your project. In most papers, the introduction includes a discussion of at least some of the previous relevant measurements, often commenting on their relative strengths and weaknesses. Since nearly all of the projects we will do have historical significance, the introduction to your report can briefly describe how the phenomena you are investigating have played a role in helping to define our present understanding of physics. And usually, an introductory section of a paper concludes with one or two sentences indicating the topics which are covered in each of the paper's subsequent sections. 

Most importantly, the introduction should present an overview of the entire project. A reader not familiar with the science will then be able to develop a broad picture of your work, before they delve into the more detailed discussion. By presenting the project's landscape early in the paper, the reader will be able to more readily comprehend the details of the project as they are subsequently offered.

You can use this file as the template of your paper. It is available for download from the Modules section of our Canvas page. The LaTeX word processing software used to create and edit this file is freely available for both PC and MAC machines at \href{https://www.overleaf.com}{Overleaf.com}. If you choose not to use Overleaf, then your word processing software should be set up to create a document with 12-point font for the main text and 0.7-inch margins on all sides. Your paper must be 5 or 6 pages in length, excluding the references and appendix.

\subsection{Important Considerations}

    As you prepare your paper, you should keep several other points in mind. Strive to achieve clarity and accuracy in your writing.

    Scientists work hard in a laboratory to make precise measurements, and they value a written report which avoids misrepresentations, ambiguities, and outright errors. 

    Imagine that your reader is another physics major not enrolled in this course. Your goal will be to produce a document which they can readily comprehend -- or, at least one which they can understand after a second reading. This can most easily be achieved by "telling the story" of your lab work. Like all stories, yours will include a beginning, a middle, and an ending. The introduction serves as the beginning of the story, followed by the development section, which provides a detailed account of your work and the final results. The ending of your story will include an interpretation of the measured data within the context you have previously established earlier in your paper.

    The first sentence of each paragraph should make a significant statement about one element of the project. The sentences which follow in each paragraph should amplify and clarify the information contained in the first. Someone should be able to read just the first sentence of each paragraph in the paper and glean a relatively thorough understanding of the project.
    
    Your written description of the methods used to obtain and analyze the data should be written in the past tense. Use the present tense when you write the conclusions. Note that your paper will be evaluated for its use of proper grammar and spelling. Please also note that in printed text the word "data" is treated as a plural noun. The singular form is "datum."